\chapter{Einleitung}

Im Kapitel Einleitung wird die Idee und persönliche Motivation, Zielsetzung und Vorgehensweise sowie der Aufbau der Arbeit beschrieben.

\section{Idee und Motivation}

Die Idee einen Algorithmus für Neural Style Transfer zu entwickeln, der auf Geräten mit leistungsarmer Hardware funktioniert, entstand durch mein persönliches Interesse an der Webentwicklung und an Machine Learning Verfahren. Bereits vor dem Studium an der HTW Berlin konnte ich Erfahrungen im Bereich der Webentwicklung/Frontend Entwicklung sammeln und habe bereits professionell in diesem Bereich gearbeitet. Im Rahmen des Hochschulstudiums lernte ich, Verfahren in den Bereichen Machine- und Deep Learning anzuwenden. Besonders der Bereich Computervision war dabei für mich interessant. Ich nutze die Bachelorarbeit dazu, diese beiden Interessensgebiete zu kombinieren und möchte die Möglichkeiten erforschen, Style Transfer Methoden auf Geräten mit leistungsarmer Hardware auszuführen.

Style Transfer ist ein interessantes Forschungsobjekt und erfordert die intensive Auseinandersetzung mit Convolutional Neural Networks \cite{lecun-gradientbased-learning-applied-1998} und dem Backpropagation Algorithmus \cite{doi:10.1162/neco.1989.1.4.541}. Außerdem müssen benutzerdefinierte Loss-Funktionen, die nicht in Problemstellungen der Bildklassifizierung verwendet werden, benutzt werden.

\section{Zielsetzung}

Ziel des Projektes ist, den Prototypen eines Softwaresystems zu erstellen, der in der Lage ist, Style Transfer auf Geräten mit leistungsarmer Hardware zu ermöglichen. Dabei sollen Mechanismen des Style Transfer untersucht und sukzessiv verbessert werden.

\section{Vorgehensweise und Aufbau der Arbeit}

Nach der Einarbeitung in die Grundlagen von Style Transfer \cite{DBLP:journals/corr/GatysEB15a, DBLP:journals/corr/JohnsonAL16} Methoden müssen entsprechende Algorithmen gefunden werden. Da es sich um die Manipulation von Bildern handelt, werden Convolutional Neural Networks \cite{lecun-gradientbased-learning-applied-1998} zum Einsatz kommen, die sich dazu eignen, Features aus Bildern zu extrahieren. Im nächsten Schritt muss ein Datensatz wie ImageNet \cite{5206848} oder COCO \cite{DBLP:journals/corr/LinMBHPRDZ14} benutzt oder in entsprechender Größe generiert werden, mit dem ein solches Netzwerk trainiert werden kann. Letztendlich muss das Netzwerk auf dem verwendeten Datensatz durch Backpropagation \cite{doi:10.1162/neco.1989.1.4.541} optimiert werden. Die Implementierung des Modells in den Prototypen eines Softwaresystems soll der Veranschaulichung der Ergebnisse dienen.

Abschließend werden umfangreiche Experimente mit verschiedenen Einstellungen und Netzwerkarchitekturen durchgeführt.
Die Ergebnisse werden miteinander verglichen und in einem Fazit kritisch betrachtet.