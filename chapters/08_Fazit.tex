\chapter{Fazit}

Das Kapitel Fazit beinhaltet eine kurze Zusammenfassung der Abschlussarbeit. Das Thema wird kritisch betrachtet und es wird auf mögliche Probleme eingangen. Letzlich wird ein Ausblick über zukünfige Entwicklungen und Möglichkeiten gewährt.

\section{Zusammenfassung}

In der Abschlussarbeit wurde sich intensiv mit Style Transfer Algorithmen beschäftig und diese auf Geräten mit leistungsarmer Hardware getestet. Anfangs wurde die Theorie zu Neuronalen Netzwerken und Style Transfer analysiert. Es wurde eine Implementierung geplant und danach mit dem Framework PyTorch umgesetzt. Unterschiedliche Netzwerke wurden erfolgreich trainiert und Stile extrahiert. Verschiedene Hyperparametereinstellungen wurden getestet und die Performanz der Netzwerke gemessen.


\section{Kritischer Rückblick}

Besonders das Training und das Durchführen der Experimente muss kritisch betrachtet werden. Es wurden die entsprechende Hyperparametereinstellungen ausführlich nur mit zwei Gemälden getestet. Fraglich ist ob diese den gleichen Effekt bei anderen Gemälden erziehlen. Deswegen wird davon ausgegangen, dass für jeden Stil die Hyperparametereinstellungen individuell getestet werden müssen.

Bei den gewählten Gemälden waren alle Netzwerkarchitekturen in der Lage den Stil zu extrahieren. Daraus kann schlussgefolgert werden das möglicherweise sogar Netzwerkarchitekturen mit weniger lernbaren Parametern, als die hier vorgeschlagenen, in der Lage sein könnten Stile aus Gemälden zu erlernen. Bei weiteren Expereminten wäre man unter Umständen in der Lage gewesen noch performantere Netzwerkarchitekturen zu finden.

Bei der Auswahl der verwendenten Gemälde musste beim Erstellen dieser Arbeit besonders vorsichtig vorgegangen werden. Die Bilder wurden aus Quellen genommen die die Verwendung für Bildungszwecke erlauben. Ein kommerzieller Einsatz wäre nicht in jedem Fall möglich. Die Verwendung von Style Transfer Methoden wirft besondere urheberrechtliche Bedenken auf. Urheberrechtlich geschützte Bilder können dazu verwendet werden Stile zu extrahieren und auf neue Bilder zu übertragen. Die in Bezug auf Urheberrecht aufkommenden Fragen müssen für den produktiven Einsatz eines Software-Systems evaluiert werden.

\section{Ausblick}

Folgend wird ein Ausblick auf Erweiterungen des Algorithmus gewährt. Dabei wird auf vorstellbare Funktionen und Einstellugen eingangen.

\subsection{Verwendung von anderer Layer-Kombination}

In dieser Arbeit wurde für die Berechnung des Style-Loss die Layer relu1\_2, relu2\_2, relu3\_3, relu4\_3 und für die Berechnung des Content-Loss der Layer relu3\_3 des \gls{vgg16}-Models verwendet. Man kann ein anderes Loss-Network und andere Kombinationen von Layern für die Berechnung der Losses benutzen. Das hätte zur Folge, dass die generierten Stile sich unterscheiden würden. Im GitHub-Repository von Logan Engstrom  \cite{engstrom2016faststyletransfer} wurde dies bereits implementiert.

\subsection{Kombination aus mehreren Stilen}
\label{sec:combination_many_styles}

Vorstellbar ist eine Vermischung von mehrern Stilen. Hierzu müsste man das Style-Loss aus die Gram-Matrizen mehrerer Gemälde kombinieren. Im Paper \cite{stanfordStyleTransfer} wurde das und andere Erweiterungen beschrieben.

\subsection{Superresolution}
\label{sec:superresolution}

Neben Style Transfer kann eine ähnliche Methodig für einen Super-Resulition-Algorithmus verwendet werden. Dieser wird ebenfalls im Paper von Johnson et al. \cite{DBLP:journals/corr/JohnsonAL16} beschrieben. Dabei muss eine Netzwerkarchitektur verwendet werden, die in der Lage ist Bilder auf eine höhere Auflösung zu skalieren. Die verwendete Loss-Funktionen wäre das bereits vorgestellte Perceptual-Loss jedoch ohne den Style-Loss-Anteil.

\subsection{Alternativen zum ResidualBlock}
\label{sec:alternatives_to_residual_block}

Performanztechnisch können weitere Experimente mit kleineren Netzwerkarchitekturen durchgeführt werden. Außerdem kann der ResidualBlock durch eine andere Art von Block ersetzt werden. Die Paper \cite{DBLP:journals/corr/HowardZCKWWAA17} und \cite{DBLP:journals/corr/abs-1801-04381} behandeln sogenannte MobileNets, welche besonders auf die Verwendung von Geräten mit leistungsarmer Hardware abziehlen. Vorstellbar wäre es die verwendenten Blöcke anstelle des ResidualBlocks zu verwenden und damit weitere Experimente durchzuführen.

\subsection{Video Stylization}
\label{sec:video_stylization}

Im Paper \cite{DBLP:journals/corr/abs-1807-01197} wird beschrieben wie Style Transfer für Videos realisiert werden kann. Dabei wird eine zusätzliche Loss-Funktion eingeführt, die ein Flickern zwischen den einzelnen Video-Frames verhindert.

\subsection{Ausführung im Webbrowser}
\label{sec:inference_in_browser}

Ein neu aufkommendes Feld ist die Realisierung von Deep Learning im Webbrowser. Tensorflow.js\footnote{\url{https://www.tensorflow.org/js}} ist ein Framework welches Modelle direkt im Browser und somit auch direkt auf beispielsweise Mobiletelefonen oder anderen Geräten mit leistungsarmer Hard ausführen kann. Ein weiteres Werkzeug ist ONNX.js\footnote{\url{https://github.com/microsoft/onnxjs}}. In PyTorch trainierte Modelle können in das ONNX-Format exportiert werden und mit ONNX.js direkt im Webbrowser ausgeführt werden. Zum Zeitpunkt der Erstellung dieser Arbeit befindet sich ONNX.js noch in einem frühen Entwicklungsstadium. Zukünfig ist ein exportieren der in dieser Arbeit erstellten Modelle vorstellbar.