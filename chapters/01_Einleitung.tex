\chapter*{Einleitung}

Die Idee einen Algorithmus für Neural Style Transfer zu entwickeln der auf Geräten mit leistungsarmer Hardware funktioniert, entstand durch mein persönliches 
Intresse an der Webentwicklung und an Machine Learning Verfahren. Bereits vor dem Studium an der HTW Berlin konnte ich Erfahrungen im Bereich der Webentwicklung/Frontend Entwicklung sammeln und habe bereits professionell in diesem Bereich gearbeitet. Im Rahmen des Hochschulstudiums lernte ich
Verfahren um Machine- und Deep Learning anzwenden. Besonders der Bereich Computervision war dabei für mich intressant. Ich nutze die Bachelorarbeit dazu diese beiden Intressengebiete zu kombinieren und möchte erforschen wie es möglich ist Style Transfer Methoden auf Geräten mit leistungsarmer Hardware auszuführen.

Die Abschlussarbeit im Bereich der Angewandte Informatik handelt darum, verschiedene Style Transfer Methoden zu implementieren, die Konzepte zu verstehen und sie auf Tauglichkeit für Systeme mit leistungsarmer Hardware zu testen. Style Transfer ist ein Bereich des Deep Learning. Es werden dabei zwei oder mehrere willkürliche Bilder zu einem neuen Bild kombiniert. Beim ersten Bild wird der Inhalt aus dem Bild herausgefiltert, z.B. aus einem Foto auf dem ein Vogel zu sehen ist. Aus dem zweiten Bild wird der Stil herausgefiltert, z.B. aus einem abstrakt gezeichneten Gemälde. Das Ergebnis ist die Mischung aus Objekt und Stil. Die Ausrichtung, ob das Ergebnis mehr Objekt oder eher Stil ist, kann durch Parameter beeinflusst werden. Es sollen unterschiedliche Algorithmen untersucht werden und die Ergebnisse miteinander veglichen werden.