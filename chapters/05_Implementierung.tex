\chapter{Implementierung}
\label{cha:implementation}

Im Bereich des Machine Learning und Deep Learning gibt es viele verschiedene Frameworks, die bei der Umsetzung und dem Training von Neuronalen Netzwerken
behilflich sind. Beispiele hierfür sind Tensorflow\footnote{\url{https://www.tensorflow.org/}} von Google, PyTorch\footnote{\url{https://pytorch.org/}} von Facebook und Apache MXNet\footnote{\url{https://mxnet.apache.org/}}. Da ich bereits im Vorfeld, während meiner Universitätsveranstaltungen, Erfahrungen mit PyTorch sammeln konnten, habe ich mich für dieses Framework entscheiden.

\section{Loss-Funktion}

Im ersten Schritt wird die Loss-Funktion implementiert, die das Kernstück des Modelbasierten und des optimisierenden Ansatz ist.
PyTorch behinhaltet bereits vortrainierte Versionen verschiedener Model Architekuren. Es wird das VGG16-Model benutzt, auf dieses
wird mit \mintinline{python}{model = torchvision.models.vgg16(pretrained=True).features} zugegriffen. Das PyTorch-Model ist dabei in die Bereiche \textit{features} und 
\textit{classifier} aufgeteilt. Es wird lediglich der Bereich \textit{features} benötigt, welcher die bereits vortrainierten Convolutional-Layer beinhaltet.
Der Bereich \textit{classifier} beinhaltet die Fully-Connected-Layer des Models, welche für die Aufgabenstellen ungenutzt bleiben.

\pagebreak

\subsection{Zugriff auf Hidden-Layer-Ergebnisse}

Um bei Pytorch auf Zwischenergebnisse der Hidden-Layer zuzugreifen, werden hinter diesen \textit{Hooks} implementiert. Das kann in eine Funktion ausgelagert werden, welche die \gls{activation_map}s während des Forwardpass durch das VVG16-Model extrahiert.

\begin{listing}[H]
\begin{minted}{python}
def extract_activation_map(image, model, layer):
    class SaveActionMap:
        def __init__(self):
            self.activation = None

        def hook(self, model, inpt, outpt):
            self.activation = outpt

    sam = SaveActionMap()

    handle = model[layer].register_forward_hook(sam.hook)
    model(image.unsqueeze(0))
    handle.remove()

    return sam.activation.squeeze()
\end{minted}
\captionof{lstlisting}{Extrahierung Activation-Maps mit PyTorch}
\end{listing}

\pagebreak

\subsection{Content-Loss}

Um das Content-Loss zu bilden wird mit der zuvor definierten Funktion auf die Zwischenergebnisse der Hidden-Layer zugegriffen.
Das Ergebnis ist das \gls{mse_loss} aller Zwischenergebnisse der Eingangsdaten (Ausgangsbild) vergliechen mit den Zwischenergebnisse der Ausgangsdaten (generierte Bilder), vgl. Kapitel \ref{sec:content_loss}.

\begin{listing}[H]
\begin{minted}{python}
def content_loss(self, y, x):
    y = extract_activation_maps(y, self.model, self.c_layers)
    x = extract_activation_maps(x, self.model, self.c_layers, detach=True)

    return sum([
        F.mse_loss(
            generated_feature,
            input_feature
        )
        for generated_feature, input_feature in zip(y, x)
    ])
\end{minted}
\captionof{lstlisting}{Berechnung des Content-Loss, vgl. Gleichung \eqref{eq:content_loss}}
\end{listing}

\subsection{Style-Loss}

Für das Style-Loss wird eine Funktion benötigt um die Gram-Matrix eines Tensors zu bilden. Die Anzahl der Dimensionen des Eingangstensors wird auf zwei Dimensionen geflächt. Diese Matrix wird seiner der von sich selbst transponierten Version multipliziert. Danach wird das Ergebnis über die das Produkt der Anzahl der Channel, Höhe und Breite des Eingangstensors normalisiert.

\begin{listing}[H]
\begin{minted}{python}
def gram_matrix(tensor):
    b, c, h, w = tensor.shape
    normalizer = c * h * w
    tensor_flat = tensor.flatten(2)

    return torch.div(
        torch.bmm(tensor_flat, tensor_flat.transpose(1, 2)),
        normalizer
    )
\end{minted}
\captionof{lstlisting}{Berechnung der Gram-Matrix, vgl. Gleichungen \eqref{eq:gram_matrix_1} u. \eqref{eq:gram_matrix_2}}
\end{listing}

\pagebreak

Die Gram-Matrix-Funktion wird dazu benutzt das Style-Loss zu berechnen. Wie beim Content-Loss wird das \gls{mse_loss} gebildet. Dieses mal werden vorher die Gram-Matrizen aller \gls{activation_map}s berechnet. Auf diese Weise wird das berechnete Ausgangsbild mit dem vorher eingestellten Style-Bild verglichen.
In dieser Implementierung wurde die Gewichtung der einzelnen Layer nicht berücksichtigt. Die Implementierung weicht daher geringfügig von der in Analyse verwendeten Gleichung \eqref{eq:style_loss} ab.

\begin{listing}[H]
\begin{minted}{python}
def style_loss(self, y):
    y = extract_activation_maps(y, self.model, self.s_layers)
    y_grams = [
        gram_matrix(row)
        for row in y
    ]

    return sum([
        F.mse_loss(
            y_gram,
            style_gram
        )
        for y_gram, style_gram in zip(y_grams, self.style_grams)
    ])
\end{minted}
\captionof{lstlisting}{Berechnung des Style-Loss, vgl. Gleichung \eqref{eq:style_loss}}
\end{listing}

\subsection{Total-Variation-Loss}

Das Total-Variation-Loss berechnet die Summe der Abweichungen eines Pixels zum nächsten Pixel eines Bildes. Dies wird für Höhe und Breite des Bildes durchgeführt und addiert.

\begin{listing}[H]
\begin{minted}{python}
def total_variation_loss(self, y):
    return torch.add(
        torch.sum(torch.abs(y[:, :, :, :-1] - y[:, :, :, 1:])),
        torch.sum(torch.abs(y[:, :, :-1, :] - y[:, :, 1:, :]))
    )
\end{minted}
\captionof{lstlisting}{Berechnung des Total-Variation-Loss, vgl. Gleichung \eqref{eq:total_variation_loss}}
\end{listing}

\subsection{Perceptual-Loss}

Das Perceptual-Loss ist das Gesamt-Loss und addiert die einzelnen Teil-Losse mit entsprechend einstellbaren Gewichtungsfaktoren.

\begin{listing}[H]
\begin{minted}{python}
def forward(self, y, x):
    content_loss = self.c_weight * self.content_loss(y, x)
    style_loss = self.s_weight * self.style_loss(y)
    total_variation_loss = self.tv_weight * self.total_variation_loss(y)

    loss = content_loss + style_loss + total_variation_loss

    return loss
\end{minted}
\captionof{lstlisting}{Berechnung des gesamten Perceptual-Loss, vgl. Gleichung \eqref{eq:perceptual_loss}}
\end{listing}

\section{Neural Style Transfer}

Die Implementation des Neural Style Transfer Algorithmus wird in einem Jupyter Notebook\footnote{\url{https://jupyter.org/}} durchgeführt. 
Dadurch können im späteren Verlauf unterschiedliche Tests durchgeführt werden, da die Hyperparameter schnell angepasst werden können.

Das Script vgl. Code snippet \ref{sec:script_neural_style_transfer} orientiert sich am zuvor erstellten Programmablaufplan vlg. Abbildung \ref{img:neural_style_pap_img}. Es bietet außerdem die Auswahl zwischen den PyTorch-Optimizern Adam und L-BFGS \cite{Liu1989}.

\begin{listing}[H]
\begin{minted}{python}
criterion = PerceptualLoss(...)
optimizer = optim.Adam([outputs])

for epoch in range(epochs):
    optimizer.zero_grad()

    loss = criterion(outputs, inputs)
    loss.backward()

    optimizer.step()    
\end{minted}
\captionof{lstlisting}{Vereinfachter Code einer Trainingsschleife in PyTorch}
\end{listing}

\section{Fast Neural Style Transfer}

Das Verfahren des Fast Neural Style funtkioniert ähnlich wie beim normalen Neural Style Transfer.
Anstatt der Pixel des Ausgangsbildes werden jedoch die Gewichte eine Neuronalen Netzwerks optimiert.
Der bestehende Code kann aus  dem vorherigen Kapitel wiederverwendet werden. 
Lediglich die Initialisierung des PyTorch Optimizers muss angepasst werden.

\begin{listing}[H]
\begin{minted}{python}
model = TransformerNet(...)
optimizer = optim.Adam(model.parameters())
\end{minted}
\captionof{lstlisting}{Optimierung der Gewichte des Netzwerk}
\end{listing}

\subsection{COCO-Datensatz}

Um das Neuronale Netzwerk zu trainieren, wird ein großer Bilddatensatz benötigt. In dieser Arbeit wird dafür der COCO-Datensatz der Firma Microsoft verwendet \cite{DBLP:journals/corr/LinMBHPRDZ14}. Es handelt sich dabei um einen öffentlichen, freiverfügbaren Datensatz vieler verschiedener Bilder, verschiedener Kategorien. Während des Trainingsverlauf werden zufällige Bilder aus dem COCO-Datensatz genutzt, 
und das Loss  (Style-, Content- und Total-Variation-Loss) über die Ausgaben, die das Netzwerk generiert, berechnet.
Mit dem berechnete Loss ist PyTorch in der Lage die Gradienten der Gewichte des Neuronalen Netzwerks zu berechnen und diese 
Schrittweise zu opitmieren.

Der COCO-Datensatz liegt in Form vieler verschiedener JPEG-Bilder vor. 
Um nicht alle Bilder gleichzeitig laden zu müssen, bietet PyTorch die Möglichkeit einen Dataloader\footnote{Weitere Information zum PyTorch-Dataloader: \url{https://pytorch.org/tutorials/beginner/data_loading_tutorial.html}}
zu nutzen, welcher die Bilder in der gewünschten Batchgröße Schritt für Schritt lädt.

\pagebreak

\begin{listing}[H]
\begin{minted}{python}
dataset = datasets.ImageFolder(
    config['dataset_path'],
    transform=transforms.Compose([
        transforms.Resize(config['content_image_size']),
        transforms.CenterCrop(config['content_image_size']),
        transforms.ToTensor()
    ])
)

dataloader = data.DataLoader(
    dataset, 
    batch_size=config['batch_size'],
    shuffle=True, 
    num_workers=8
)
\end{minted}
\captionof{lstlisting}{Der PyTorch-Dataloader lädt Bilder in der gewünschten Batchgröße}
\end{listing}

Der Dataloader lädt alle Bilder in allen Unterordnern innerhalbes des konfigurierten Pfades.

\subsection{Netzwerkarchitektur}
\label{sec:network_architecture}

Die Architektur des Neuronalen Netzes spielt eine wichtige Rolle. Sie muss in der Lage die Eigenschaften eines Stils möglichst gut zu erlernen und dabei Performant bleiben. Dabei wurde die in Kapitel \ref{sec:fast_neural_style_transfer} vorgestellte Netzwerkarchitektur implementiert.
Dies geschieht in PyTorch in Form einer Klasse, welche von \mintinline{python}{Module} erbt.

Um in Kapitel \ref{cha:tests} das Testen unterschiedlicher Netzwerkarchitekturen zu ermöglichen, wird die Klasse des Transformer Networks möglichst dynamisch implementiert. Einstellbar sind dabei folgende Werte, die im weiteren Verlauf erklärt werden:

\begin{itemize}
    \item channel\_multiplier
    \item bottleneck\_size
    \item bottleneck\_type
    \item final\_activation\_fn
    \item intermediate\_activation\_fn
\end{itemize}

\subsection{Aktivierungsfunktionen}

Benötigt wird eine Funktion, die entsprechende PyTorch-Aktivierungsfunktionen zurück gibt. Hardtanh und Sigmoid eigenen sich als finale Aktivierungsfunktionen des Netzwerks, da Sie Werte zwischen $ 0.0 $ und $ 1.0 $ zurückliefern. Das eignet sich für die Generierung von Bildern, welche in PyTorch ebenfals Pixelwerte zwischen $ 0.0 $ und 1.0 besitzen. Zu beachten ist das die Hardtanh-Funktion mit den Parametern \mintinline{python}{min_val=0.0} und \mintinline{python}{max_val=1.0} initialisiert wird, welche den Wertebereich verschieben. Mit den Standardeinstellungen liegt der Wertebereich der Hardtanh-Funktion zwischen $ -1.0 $ und $ 1.0 $. Die Parameter \mintinline{python}{final_activation_fn} und \mintinline{python}{intermediate_activation_fn} werden dabei auf die entsprechenden PyTorch-Aktivierungsfunktionen überführt.

\begin{listing}[H]
\begin{minted}{python}
def get_activation_fn(name):
    activation_fn_map = {
        'ELU': lambda: nn.ELU(),
        'ReLU': lambda: nn.ReLU(),
        'RReLU': lambda: nn.RReLU(),
        'PReLU': lambda: nn.PReLU(),
        'SELU': lambda: nn.SELU(),
        'CELU': lambda: nn.CELU(),
        'ReLU6': lambda: nn.ReLU6(),
        'Hardtanh': lambda: nn.Hardtanh(min_val=0.0, max_val=1.0),
        'Sigmoid': lambda: nn.Sigmoid()
    }

    return activation_fn_map[name]()
\end{minted}
\captionof{lstlisting}{Map der Aktivierungsfunktionen}
\end{listing}

\subsection{ConvBlock}

Desweiteren müssen die in Kapitel \ref{sec:aufbau} konzipierten Layer-Blöcke implementiert werden.
Der \textit{ConvBlock} besteht aus den PyTorch-Layern \mintinline{python}{Conv2d}, \mintinline{python}{InstanceNorm2d} und der gewählten Aktivierungsfunktion.
\mintinline{python}{nn.InstanceNorm2d} wurde auf Grund des Papers \cite{DBLP:journals/corr/UlyanovVL16} gewählt und dort näher in Bezug auf die Verwendung mit Style-Transfer-Methoden erläutert.

\pagebreak

Die Parameter \mintinline{python}{in_channels} und \mintinline{python}{out_channels} ergeben sich aus dem Parameter \mintinline{python}{channel_multiplier} multipliziert mit einem festen Wert, der je nach Layer $ 1 $, $ 2 $ oder $ 4 $ beträgt.

\begin{listing}[H]
\begin{minted}{python}
class ConvBlock(nn.Module):
    def __init__(self, in_channels, out_channels, kernel_size=3, stride=1, activation_fn='PReLU'):
        super(ConvBlock, self).__init__()

        self.conv = nn.Conv2d(in_channels, out_channels, kernel_size, stride)
        self.norm = nn.InstanceNorm2d(out_channels, affine=True)
        self.activation_fn = get_activation_fn(activation_fn)

    def forward(self, x):
        returnself.activation_fn(self.norm(self.conv(x)))
\end{minted}
\captionof{lstlisting}{Aufbau des ConvBlock}
\end{listing}

\subsection{ResidualBlock}

Der ResidualBlock, vgl. \ref{img:residual_block_img}, besteht aus zwei aufeinanderfolgenden \mintinline{python}{Conv2d}-Layern. Wie beim vorherigen ConvBlock werden die Daten über \mintinline{python}{InstanceNorm2d} normalisiert, anschließend wird die Aktivierungsfunktion angewendet. Die Besonderheit ist das die Eingangdaten kopiert werden und danach mit den berechneten Daten der Convolutional-Layer addiert werden. Durch \mintinline{python}{Conv2d} verkleinert sich die Dimension der Daten, vgl. \ref{sec:conv_networks}, welches durch die \mintinline{python}{ReflectionPad2d}-Layer wieder ausgeglichen wird. Eingangsdaten und berechnete Daten müssen die gleichen Dimensionen aufweisen um addiert werden zu können. Die Parameter \mintinline{python}{in_channels} und \mintinline{python}{out_channels} ergeben sich aus \mintinline{python}{channel_multiplier} multipliziert mit $ 4 $.


\begin{listing}[H]
\begin{minted}{python}
class ResidualBlock(nn.Module):
    def __init__(self, in_channels, out_channels, inner_channels=None, kernel_size=3, activation_fn='PReLU'):
        super(ResidualBlock, self).__init__()
        inner_channels = inner_channels if inner_channels else in_channels
        self.conv1 = nn.Conv2d(in_channels, inner_channels, kernel_size)
        self.conv2 = nn.Conv2d(inner_channels, out_channels, kernel_size)
        self.pad1 = nn.ReflectionPad2d(padding=kernel_size // 2)
        self.pad2 = nn.ReflectionPad2d(padding=kernel_size // 2)
        self.norm1 = nn.InstanceNorm2d(inner_channels, affine=True)
        self.norm2 = nn.InstanceNorm2d(out_channels, affine=True)
        self.activation_fn = get_activation_fn(activation_fn)

    def forward(self, x):
        identity = x

        x = self.activation_fn(self.norm1(self.conv1(self.pad1(x))))
        x = self.norm2(self.conv2(self.pad2(x)))

        return x + identity
\end{minted}
\captionof{lstlisting}{Aufbau des ResidualBlock}
\end{listing}

\subsection{UpSampleBlock}

Der UpSampleBlock schaltet lediglich dem ConvBlock ein Upsampling-Verfahren vor, um das Bild um einen entsprechenden Faktor zu vergrößern.

In dieser Arbeit wird Nearest Neighbor Interpolation (ggf Paper hier einfügen) verwendet.


\begin{listing}[H]
\begin{minted}{python}
class UpSampleBlock(nn.Module):
    def __init__(self, in_channels, out_channels, kernel_size=3, stride=1, scale_factor=None,
                 activation_fn='Hardtanh'):
        super(UpSampleBlock, self).__init__()

        self.scale_factor = scale_factor

        self.conv = ConvBlock(
            in_channels=in_channels,
            out_channels=out_channels,
            kernel_size=kernel_size,
            stride=stride,
            activation_fn=activation_fn
        )

    def forward(self, x):
        if self.scale_factor:
            x = F.interpolate(x, mode='nearest', scale_factor=self.scale_factor)

        return self.conv(x)
\end{minted}
\captionof{lstlisting}{Aufbau des UpSampleBlock}
\end{listing}

\subsection{TransformerNet}

Neben den Layer-Blöcke wurde noch die eigentliche Netzwerk Architektur implementiert. Die Implementierung nimmt die in \ref{sec:network_architecture} beschriebenen Parameter im Konstruktor entgegen um eine entsprechende Netzwerk-Architektur zu generieren. Neben den bereits beschriebenen Einstellungen gibt \mintinline{python}{bottleneck_size} die Anzahl der hintereinander geschalteten ResidualBlocks an. Der \mintinline{python}{bottleneck_type} ist für zukünftige Erweiterungen vorgesehen, damit der ResidualBlock durch einen anders aufgebauten Block ersetzt werden kann. Außerdem wurde ein anfängliches Padding eingefügt um die ursprüngliche Bildgröße beizubehalten.

\begin{listing}[H]
\begin{minted}{python}
class TransformerNet(nn.Module):
    def __init__(
            self,
            channel_multiplier=32,
            bottleneck_size=5,
            bottleneck_type=BottleneckType.RESIDUAL_BLOCK,
            final_activation_fn='Hardtanh',
            intermediate_activation_fn='PReLU'
    ):
        super(TransformerNet, self).__init__()

        self.pad = nn.ReflectionPad2d(padding=20)

        # Geneierung der Architektur

    def forward(self, x):
        x = self.pad(x)
        x = self.down3(self.down2(self.down1(x)))
        x = self.bottleneck(x)
        x = self.up3(self.up2(self.up1(x)))

        return x
\end{minted}
\captionof{lstlisting}{Aufbau des TransformerNet, kompletter Code: \ref{sec:transformer_net_full}}
\end{listing}

\pagebreak

\subsection{Weitere Implementierungen}

Zusätzlich zu den bereits vorgestellten Implementierungen wurden viele Hilfsskripte und Jupyter Notebooks zum Testen erstellt.
Außerdem wurde eine Trainer-Klasse entworfen die in der Lage ist Einstellungen für Trainingsläufe aus einer Konfiguration Datei zu lesen und somit 
ein aufeinanderfolgendes Training mehrerer Modelle zu ermöglichen. Auf diese Arte ist es möglich viele unterschiedliche Parametereinstellungen zu testen.

Zur Realisierung eines Prototypens wurde eine JSON-API mit dem Framworks Flask\footnote{Flask: \url{http://flask.pocoo.org/}} erstellt. Die API stellt einen Endpunkt zur verfügung welche den Style Transfer mit einem gewünschten Modell durchführt. Sie außerdem in einen Docker\footnote{Docker: \url{https://www.docker.com/}}-Container überführt um Hosting auf unterschiedlichen Geräten zu erleichtern.

Um die Ergebnisse zu visualisieren wurde eine Frontend-Applikation mit dem Framework Angular\footnote{Angular: \url{https://angular.io/}} erstellt. Diese kann Bilder über die Webcam eines Handys oder eines Computers aufnehmen und Sie an die Flask-API schicken, welche das geschossene Bild in einen vorher ausgewählten Stil überführt.

Skripte, Frontend-Applikationen und JSON-API sind nicht Bestandteil dieser Arbeit und befinden sich in einem protypischen Zustand. Der komplette Code zu diesem Projekt ist unter \url{https://github.com/christophstach/style-transfer-project} öffentlich einsehbar. Der Code des Frontends unter \url{https://github.com/christophstach/style-transfer-frontend}. Ein gehosteter Prototyp zum Testen des Style Transfers ist unter \url{https://stylized.christophstach.me/} verfügbar.