\chapter*{Danksagung}

Ich danke allen Personen, die mich bei der Erstellung dieser Abschlussarbeit unterstützt haben. Dabei gilt besonderer Dank meinen Eltern, Annegret Stach und Rüdiger Stach sowie meiner Schwester Pauline Stach, die mich während der gesamten Zeit meines Studiums unterstützt haben.

Weitere Danksagungen gehen an meine Freunde sowie meine Betreuer Frau Prof. Dr. Christin Schmidt und Herrn Patrick Baumann, die mir bei der Erstellung dieser Arbeit mit Rat und technischen Ressourcen zur Seite standen.

\pagebreak

\chapter*{Abstract}

Die Abschlussarbeit im Bereich der angewandten Informatik handelt davon, verschiedene Style Transfer Methoden zu implementieren, die Konzepte zu verstehen und sie auf Tauglichkeit für Systeme mit leistungsarmer Hardware zu testen. Style Transfer ist ein Bereich des Deep Learning. Es werden dabei zwei oder mehrere willkürliche Bilder zu einem neuen Bild kombiniert. Beim ersten Bild wird der Inhalt aus dem Bild herausgefiltert, z.B. aus einem Foto, auf dem ein Vogel zu sehen ist. Aus dem zweiten Bild wird der Stil herausgefiltert, z.B. aus einem abstrakt gezeichneten Gemälde. Das Ergebnis ist die Mischung aus Objekt und Stil. Die Ausrichtung, ob das Ergebnis mehr Objekt oder eher Stil ist, kann durch Parameter beeinflusst werden. Es werden unterschiedliche Algorithmen sowie ihre Hyperparameter untersucht. Anschließend wird ihre Performanz gemessen und die Ergebnisse miteinander verglichen.